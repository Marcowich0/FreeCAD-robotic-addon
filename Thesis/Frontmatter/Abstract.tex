\chapter*{Abstract}
\addcontentsline{toc}{chapter}{Abstract}

With the increasing importance of wind energy, optimisation of such production facilities is likewise becoming more relevant in recent times. A common problem one might encounter in large wind farms is the consequence of turbines standing in each other's wake. This paper aims to investigate the behaviour of downstream turbines with a combination of CFD and aeroelastic calculations. Furthermore, the possibility of strategically reducing the power of specific turbines to reduce such phenomenon will be explored. Since the flow simulations are costly, a surrogate will be trained to predict behaviour, which can then be applied for later optimisations. Lastly, it is investigated how yawing and pitching might be used to maximise the total power output without increasing the loads unnecessarily. This showed great promise, with a maximum increase of approximately seven to eight percent depending on the spacing between the turbines. Unfortunately, no certain conclusions could be made due to the high uncertainties, but the analysis gave great insight into how these could be improved. Ultimately it was concluded that additional data was needed to get accurate results and that the thesis primarily should be viewed as a basis for further research.