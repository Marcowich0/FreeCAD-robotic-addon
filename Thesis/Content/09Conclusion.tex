\chapter{Conclusion}

Chapter \ref{chap:introduction} illuminated why wind energy is important in recent times based on a report from the International Energy Agency \cite{powerconsumption}. It also introduced the problem of wakes in wind farms and suggested how WFFC, as presented in this thesis, can be relevant to alleviate this problem. The methodologies of how we might investigate such optimisations were then presented mathematically to lay a foundation of understanding when applying the methods. In practice, these methods are primarily implemented through scripts, which generate the data necessary for the analysis. Building an underlying sense of how the turbines behave is useful for intuitively validating the results, and a single turbine was therefore considered under different conditions to learn about its preferred operation. The general control was analysed by applying a uniform inflow with different velocities, and dimensionless numbers were applied to evaluate how much changing control parameters affect the individual turbine. 

The simulation data is the foundation of the analysis, so it is important to clarify which data is applied. The different cases for the simulations were presented in chapter \ref{chap:data}, where it was also introduced how it would be simplified and represented through a Weibull distribution. 

Based on the Weibull distribution, it was possible to train the surrogate, which generally performed well, considering the limited training data. Although it was not perfect, especially for the flap loads, this uncertainty seemed relatively low compared to the natural spread of the distribution. Generally, the analysis showed great promise of increasing power, even for lower load constraints, but was somewhat shadowed by the high uncertainties. From the discussion, we investigated a concrete time series, which behaved exactly as we would expect. Here we saw that it was indeed possible to achieve a slight improvement in total power by yawing the turbines. Additionally, the results were compared to a similar paper, which was slightly more conservative but got results of a similar proportion. 

Ultimately we must conclude that there is no evidence for the analysis to be incorrect but that it is difficult to prove a certain increase due to the high uncertainties. Though this thesis may not have proven direct evidence of a beneficial optimisation, it showed a clear possibility of such and proposed multiple ways of improving the methodology and, thereby, the accuracy of the results. First, it is believed that additional CDF simulations would be necessary to determine a maximum, primarily, but not exclusively, for short-distance optimisations. Most importantly, however, should the CDF simulations be extended to prolong each period, to get a more accurate average and reduce the spread of the Weibull distributions. Such a refinement might not alter the results, but it should lower the uncertainty and strengthen the statistical argument for the increase. Generally, this thesis should not be seen as a direct application of wind farm flow control but rather as an introduction to the methodology and a basis for further research. 